\documentclass[USletter,12pt]{article}
\usepackage[margin=0.75in]{geometry}
\usepackage{enumerate}
\usepackage[pdftex]{graphicx}
\usepackage{hyperref}
\hypersetup{colorlinks, citecolor=black, filecolor=black, linkcolor=black, urlcolor=black}
\newcommand{\HRule}{\rule{\linewidth}{0.5mm}}
\begin{document}

\begin{titlepage}
\begin{center}

\includegraphics[width=0.4\textwidth]
{./Duncan_Crest}~\\[1cm]

% Title
\HRule \\[0.4cm]
{ \huge \bfseries Anne and Charles Duncan College Constitution}\\[0.2cm]
\HRule \\[0.4cm]

\textsc{\LARGE William Marsh Rice University}\\[1.5cm]

\vfill

% Bottom of the page
%{\large Last Updated: \today}
{\large Ratified: February 2nd, 2014}\\
{\large Last Amended: January 21, 2015}

\end{center}
\end{titlepage}




{\small \tableofcontents}



\newpage


%*****Introduction*****
%***********************
\section{Introduction}
%***********************
%*****Introduction*****



%***Name ***
\subsection{Name}


The name of the college is Anne and Charles Duncan College of Rice University.  It is so named in honor of Anne and Charles Duncan for their unparalleled contributions and services to Rice University.  Hereafter in this document it will be referred to as Duncan, Duncan College, or the college.


%***Mission Statement***
\subsection{Foundations of the College}


\subsubsection{Mission Statement}
The purpose of Duncan College and the government and members thereof is to create and maintain an atmosphere that encourages the social, personal, and intellectual growth and wellbeing of its members and to provide a setting that facilitates the sharing of ideas and enthusiasm with their peers and with the faculty and staff of Rice University.  The college shall develop this atmosphere through fellowship, bonding, and community born of living, dining, and engaging in the tradition of college life together.

\subsubsection{Motto}
The motto of Duncan College is ``Classis et Germanitas" (Class and Brotherhood).  The origins of the motto come from Duncan's first Beer Bike in 2010 when the college rallied around a fallen biker, causing an emotional Luis Duno-Gottberg (the first Master of Duncan College) to state ``Somos Equipo, Somos Familia" (we are a team, we are a family) in his native Spanish.  The motto serves to remind us of the bond we share as members of the college.

\subsubsection{Duncan College Crest}
The crest of Duncan, shown on the cover page, shall serve as the symbol of the college and as such is imbued with symbolism of the college and its namesake.  The oak tree that serves as the predominant image of the crest represents the commitment of Duncan to being an environmentally friendly college, inspired by Charles Duncan's term as United States Secretary of Energy.  The oak tree, noted for its strength and endurance, also represents the persistence and vitality of the college.  Additionally, the oak tree is representative of the Duncan Oaks, a collection of oak trees named after Anne and Charles Duncan located in Founder's Court in front of Lovett Hall.  The tree, along with the rest of the crest, is in the official colors of Duncan College: Gold, Forest Green, and Ivory.  Within the crest, the building of Duncan College is shown from the south.  The sun above the building is in the east, thereby making it a rising sun and symbolizing the youth, rising nature, and bright future of the college.  The D in the center of the crest represents Duncan.  The Duncan Motto is inscribed in the trunk of the tree.  The owl situated at the top of the crest serves to remind everyone that individuals in Duncan are Rice students above all else.

\subsubsection{Guiding Principles}
The Guiding Principles of the Duncan College Government are accountability, excellence, fairness, integrity, practicality, and transparency. At its core, the Duncan College Government functions and acts according to these principles. These attributes set the standard for evaluating and creating practices and procedures, and these shared values guide all individual actions.

\subsubsection{Intent of the Constitution}
The intent of the Duncan College Constitution is to provide a framework for the structure and operation of the Duncan Government.  This is done in order that the officers of the Government can best fulfill their roles in the service of Duncan. 
The Duncan College Constitution is not written to be an exhaustive document.  It is written to show how we apply our Guiding Principles to our government.  In the case that the government is faced with a situation where guidance from the Constitution is missing or incomplete, then the LVP may make a determination to fill that gap as outlined in ``Procedural Overview - Head of Procedure."  If the Duncan Forum would like to override that determination, they may in accordance with ``Functioning of Forum - Voting Power."

%***Membership***
\subsection{Membership}


\subsubsection{Determining Membership}

New membership of Duncan College shall be determined by the Office of Undergraduates from entering new undergraduate students of Rice University and, once determined, shall not be subject to change except by the consensus of the Duncan Masters and the Office of Undergraduates.  Membership in the college is a lifelong privilege.


\subsubsection{Implications of Membership}

Only members of Duncan College currently enrolled in Rice University are allowed to participate in the proceedings of the Duncan Government.  This includes, but is not limited to: running for or serving in elected or appointed office, entering Room Jack, entering Room Draw, entering Parking Jack, submitting petitions for impeachment, and acting as a proxy.

%***University Rules and Regulations***
\subsection{University Rules and Regulations}

\subsubsection{University}
The rules and regulations of Rice University apply in full to Duncan College.  In the event of a conflict with the Duncan College Constitution, the rules of the University shall supersede those of the College.

\subsubsection{Duncan College Master}
The Duncan College Master has the power to override the Duncan Constitution.  They can overturn any decision made by any recognized formal entity of the college. This includes, but is not limited to: the Executive Committee, any member of the Executive Committee, Forum, the Duncan Court, a representative, or committee.  Such an override will be disclosed to Forum as fully as possible while respecting any confidentiality concerns related to an individual member of the college.




%*****Governmental Structure*****
%***********************************
\section{Governmental Structure}
%***********************************
%*****Governmental Structure*****



%***The Duncan Forum***
\subsection{The Duncan Forum}

\subsubsection{Membership}
All students who fulfill the requirements of membership to Duncan College are also by definition members of the Duncan Forum.

\subsubsection{Officers of the Duncan Forum}
The Officers of the Duncan Forum are the President, the Chief Justice, two Vice Presidents, the Legislative Vice President, a First Treasurer, a Second Treasurer, two Secretaries, the Student Association Senator, and eight Class Representatives.  If any officer of the Duncan Forum cannot attend a meeting of the Duncan Forum, it is his or her responsibility to designate a proxy for that meeting.

\subsubsection{Appointment}
All officers of the Duncan Forum are either elected by the entire college or by the class they represent in accordance with the Duncan Election Code of Conduct.

\subsubsection{Chair of the Duncan Forum}
The President is the chair of the Duncan Forum.

\subsubsection{Moderator of the Duncan Forum}
The President is the moderator of Forum.  In the case that the President would like to participate in the discussion, the President must appoint a temporary moderator.  The required order of succession of moderator is President, LVP, and CJ.  If none of those three people is present and able to act as moderator, then the discussion is tabled until such a time that one of them can fulfill the role.

\subsubsection{Meetings}
Meetings of the Duncan Forum must be open to all Duncan students with sufficient time allotted for any member of the community to voice questions, comments, or concerns. Meetings must be held regularly and frequently enough to properly address the matters and issues facing the college.

\subsubsection{Powers and Duties of the Class Representatives}
\begin{enumerate}[(a)]
\item The Class Representatives are responsible for voicing the opinions of their class in all matters of discussion.  They are to act as liaisons between the members of their class and the Duncan Government.
\item In matters of voting, the Class Representatives are to vote on behalf of their class, representing the will and self-interest of the class.
\item Each class elects two Representatives from within itself in an election restricted to only members of that class.  For the purposes of voter class designation, the class will be designated in the same manner as done in Room Jack and Room Draw.
\end{enumerate}



%***Executive Committee***
\subsection{Executive Committee}


\subsubsection{Membership}
The Duncan Executive Committee (EC) consists of the President, the Chief Justice, two Vice Presidents, the Legislative Vice President, a First Treasurer, a Second Treasurer, two Secretaries, and the Student Association Senator.

\subsubsection{Appointment}
All members of the EC are elected by the entire college in accordance with the Duncan Election Code of Conduct.

\subsubsection{Holding of Multiple Offices}
No member of the Executive Committee may simultaneously hold another Duncan Government position.  The only exception to this is EC offices other than the President and Chief Justice may serve as Orientation Week Coordinator.

\subsubsection{Going Abroad}
If a person is planning to study abroad, he or she is encouraged not to run for an office in which his or her term would overlap with being abroad.  In the event that a person already in office decides to study abroad, then prior to his or her departure he or she is to submit a resignation.

\subsubsection{Table Overview of Executive Committee}
\begin{tabular}{|l| l| l| l| l|}
\hline
	\bfseries{Positions}&\bfseries{Forum Vote}&\bfseries{EC Vote}&\bfseries{Bump Exempt}&\bfseries{Reports To:}\\
\hline\hline
	President&Tie breaker only&Tie breaker only&Yes, must live on campus&-\\
\hline
	Chief Justice&No&Yes&Yes, must live on campus&-\\
\hline
	Vice Presidents&Yes, 1 each&Yes, 1 each&Yes&President\\
\hline
	Legislative VP&No&No&Yes&-\\
\hline
	Treasurers&Yes, 1 together&Yes, 1 together&No&President\\
\hline
	Secretaries&Yes, 1 together&Yes, 1 together&No&President\\
\hline
	SA Senator&No&Yes&No&-\\
\hline
\end{tabular}

\subsubsection{Powers and Duties of the President}
\begin{enumerate}[(a)]
\item The Duncan College President is the highest office at Duncan and has the responsibility to oversee all governmental functions at Duncan.  This must be done in accordance with the Duncan Constitution and without exerting undue power over the responsibilities of the other offices.
\item The President is responsible for representing the college to the University and the Student Association. The President is responsible for handling all obligations expected of the position by the University.  
\item The President is responsible for running meetings of the Duncan Forum such that the Duncan government can regularly and openly interact with the Duncan community.  The President has the power to set the agenda for Forum but may not suppress topics that the community wants to discuss.
\item The President is the chair of the Duncan Executive Committee and is expected to hold regular meetings of the Executive Committee such that all members of the Executive Committee are apprised of the proceedings of the University and Duncan College.
\item The President is the default moderator of Forum meetings and is therefore expected to remain unbiased in discussions unless he or she passes off the duties of moderator.
\item In the event that one of the two-person offices (VPs, Secretaries, or Treasurers) has a disagreement they cannot resolve, the President has the power to resolve the dispute.  This does not apply to the casting of votes.
\item The President has the power to make a declaration of war against any other college for the reasons of being ``lame," ``meh...," or ``not really that fun."
\item The President is responsible for looking after the Duncan flag and the Presidential Staff.  Additionally, the President is responsible for making sure that the names on the staff are up to date.
\end{enumerate}

\subsubsection{Powers and Duties of the Chief Justice}
\begin{enumerate}[(a)]
\item The Chief Justice (CJ) is responsible for overseeing all judiciary matters that the University delegates to the colleges.
\item The CJ is the head of the Duncan Court and is responsible for upholding the Duncan Code of Conduct and the Court procedures.
\item The CJ must appoint Associate Justices to aid him or her in upholding the Code of Conduct.
\item The CJ has the power to appoint an Acting CJ for a temporary period of time.  An Acting CJ has none of the constitutional powers of the CJ.  An Acting CJ is simply the temporary primary point of contact.  Upon his or her return, the CJ is expected to resume the role and relieve the Acting CJ of all relevant responsibilities. Any issue that arose under the Acting CJ should be addressed jointly by the CJ and the Acting CJ, with all power and discretion given to the CJ. The CJ is expected to notify the college when an Acting CJ is appointed and when that appointment will expire.
\item If the CJ is unreachable and has not appointed an Acting CJ, then the President becomes the Acting CJ.
\item Without a trial, the CJ only has the power to assign penalties for actions explicitly specified in the Code of Conduct.  If the accused finds issue with the penalty, he or she may call for a trial.
\end{enumerate}

\subsubsection{Powers and Duties of the Vice Presidents}
\begin{enumerate}[(a)]
\item There are two Vice Presidents (VPs) elected as the top two winners of an election in accordance with the Election Code of Conduct.
\item The Vice Presidents are responsible for selecting representatives where applicable and all heads of standing committees.
\item The Vice Presidents are responsible for overseeing all standing committees, all applicable representatives, and the First Technical Director.  The Vice Presidents have final authority over all offices they select and the First Technical Director.
\item The Vice Presidents are responsible for overseeing the scheduling of committee events and all other events involving Duncan to identify and minimize conflicts.
\item The Vice Presidents are responsible for managing the organization and access to the storage spaces designated for Duncan's committees.
\item Before the end of the academic year, the Vice Presidents must put forth a written proposal defining all of the committees and representatives in accordance with the ``Committees and Representatives" section.
\item The Vice Presidents are responsible for dividing up the responsibilities of the office between themselves and for structuring the hierarchy beneath them.  The Vice Presidents must structure the hierarchy such that all committees and representatives directly report to a Vice President.
\end{enumerate}

\subsubsection{Powers and Duties of the Legislative Vice President}
\begin{enumerate}[(a)]
\item The LVP is responsible for all legislative matters.  Legislative matters include, but are not limited to: upholding, interpreting, and carrying out the processes of the Constitution, updating the Constitution when changes have been passed, and educating the rest of Duncan about the Constitution.
\item The Legislative Vice President (LVP) oversees all governmental voting within the college.  This includes: votes of the EC, votes of Forum, and the voting for all generally elected positions.  The LVP is allowed to participate in discussions.  However, once the LVP assumes the role of moderating a vote, he or she is to remain impartial.
\item The LVP is responsible for administering Room Jack, Room Draw, and Parking Jack.
\item The LVP is the primary administrator of Freshmen Service Points and is expected to collaborate with the VPs and committee heads to make sure that the needs of the college are appropriately met.
\item Due to the nature of the position, the LVP is expected to maintain impartiality.  If there is a conflict of interest involving the LVP, then the LVP has one of two options:
	\begin{enumerate}[(I)]
	\item Bring impartial people into the process to oversee the procedure.
	\item Recuse themselves and appoint others to perform the process in question.  This appointment will serve only until the procedure has been carried out.
	\end{enumerate}
\item The LVP is responsible for ensuring that the procedure of the Duncan Court is followed and that the rights of the accused are upheld.  The LVP may act as a counsel for the accused if the accused so chooses.
\end{enumerate}

\subsubsection{Powers and Duties of the Treasurers}
\begin{enumerate}[(a)]
\item The Treasurers are responsible for creating and managing the Duncan College budget passed by Forum.
\item The Treasurers are responsible for keeping track of the spending of college funds and making sure the passed budget is enforced.
\item The Treasurers are responsible for modifying the budget if such a change is passed by Forum.
\item In the event that some person or committee overspends their budget, the Treasurers have the power to levy a fine against the signatory for the amount spent over budget.  
\item Office of the Second Treasurer
	\begin{enumerate}[(I)]
	\item To be eligible to run for Second Treasurer, one must be able to serve for two terms.
	\item For all reasons, official and otherwise, the Second Treasurer will be considered an equal partner with the First Treasurer.
	\end{enumerate}
\item Office of the First Treasurer
	\begin{enumerate}[(I)]
	\item A Second Treasurer becomes First Treasurer by passing a vote of confidence by the outgoing EC in the month before the first round elections.
	\item In the event that there is no Second Treasurer able to fill the role of First Treasurer, then any member of Duncan can run for First Treasurer.
	\end{enumerate}
\item The term limit on Treasurers is a maximum of one year as Second Treasurer and a maximum of one year as First Treasurer. 
\end{enumerate}

\subsubsection{Powers and Duties of the Secretaries}
\begin{enumerate}[(a)]
\item The primary responsibility of the Secretaries is to handle communications for the college.  This includes, but is not limited to: overseeing college e-mails, overseeing all printed signs hung throughout the college, and taking Forum and Executive Committee meeting minutes and providing them to the proper recipients.
\item The Secretaries are responsible for designating reservable rooms within the college and overseeing their reservations.
\item The Secretaries must run for their office in teams of two.  Once elected, there is no seniority or official distinction between the secretaries.  They are responsible for dividing up the responsibilities of the office between themselves.
\end{enumerate}

\subsubsection{Powers and Duties of the Student Association Senator}
\begin{enumerate}[(a)]
\item The Student Association (SA) Senator is responsible for representing Duncan and its interests to the Student Association.
\item The SA Senator is responsible for keeping the college apprised of the matters of the Student Association and getting feedback from the college so that he or she may better represent the college.
\item The SA Senator is responsible for fulfilling all duties specified by the SA.
\end{enumerate}

%***The Duncan Court***
\subsection{The Duncan Court}
\subsubsection{Membership}
The Court shall consist of the Chief Justice and the eight Associate Justices (AJs).

\subsubsection{Jurisdiction of the Court}
\begin{enumerate}[(a)]
\item The Court has jurisdiction only on types of infractions which the Rice University Code of Student Conduct allows college courts to hear.  
\item If University rules allow a case to be heard by the Duncan Court, then the court has jurisdiction over any student involved in a suspected infraction of the Duncan College Code of Conduct and/or Rice University rule.
\item The Court may refer cases to the University Court by a majority vote.
\end{enumerate}

\subsubsection{Powers and Duties of the Duncan Court}
\begin{enumerate}[(a)]
\item The Court has the power to investigate, to hold hearings, and to determine decisions and sanctions.
\item The Court has the duty to investigate grievances and gather relevant evidence for hearings.
\item The Court has the duty to inform the accused of his or her rights.
\end{enumerate}

\subsubsection{Selection of the Associate Justices}
\begin{enumerate}[(a)]
\item The eight Associate Justices (AJs) for each academic year are selected by the CJ after an application process.  The CJ is strongly encouraged to select two AJs from each of the four classes.  There are to be six returning student AJs to be selected before the end of classes during the academic year prior to the one during which they will serve.  Two additional AJs are to be selected quickly after the start of the academic year in which they are to serve, with a strong preference that they be new students.  All AJ selections of the CJ must be approved by the College Master before they are considered installed.
\item In the event a Justice resigns or is impeached from his or her office, the vacancy is filled in the same application and approval process for the standard selection of AJs.
\end{enumerate}

\subsubsection{Powers and Duties of the Associate Justices}
The Associate Justices are responsible for assisting the Chief Justice in upholding the Duncan Code of Conduct.  They are selected as involved members of the community and are therefore expected to act proactively to avert situations that are dangerous or otherwise in violation of the Code of Conduct.  If a violation does occur then the AJs are expected to be the eyes and ears of the CJ and report pertinent information back to the CJ.
\begin{enumerate}[(a)]
\item The AJs are expected to assist with the security at public Duncan events at the request of the CJ.
\item The CJ should specify to what extent the AJs should report and themselves enforce violations to the Code of Conduct.
\item In the event that an AJ witnesses a violation of the Duncan Code of Conduct, he or she has the responsibility to inform the violator/s, assuming he or she judges that it is safe to do so.
\item The AJs are expected to live up to the Duncan Code of Conduct in the same manner as any other member of Duncan.
\end{enumerate}


%***Committees and Representatives***
\subsection{Committees and Representatives}

After Changeover but before the end of the academic year, the two Vice Presidents shall create a written proposal and present it to Forum which defines all of the standing committees and representatives they want to create and collectively oversee in the coming academic year.  Any piece of that document can be overturned by a vote of the EC or Forum.  The VPs are expected to recuse themselves from such a vote.

\subsubsection{Standing Committees}
\begin{enumerate}[(a)]
\item No committee can have any power that overlaps with any office that is constitutionally defined.
\item The VPs have the power to appoint committee heads to their committees.
\item The committee heads shall select committee members to form an appropriately sized committee through an application process.
\item At the end of the academic year, all standing committees are automatically dissolved.
\end{enumerate}

\subsubsection{Working Committees}
\begin{enumerate}[(a)]
\item The President has the power to create a Presidential Working Committee without a vote of EC or Forum.  However, the President must promptly announce the committee to Forum.
	\begin{enumerate}[(I)]
	\item Upon the creation of the Presidential Working Committee, the President must appoint committee heads and define how members will be chosen.
	\item Presidential Working Committees may only act in an advisory capacity.  They shall have no explicit powers.
	\item All Presidential Working Committees are dissolved at the Changeover at end of creator's presidential term.
	\end{enumerate}
\item Any voting member of Forum may motion to create a Forum Working Committee.
	\begin{enumerate}[(I)]
	\item Forum Working Committees can be given explicit power as long as they do not overlap with any existing committee or constitutionally defined office.
	\item Forum Working Committees can be defined to dissolve either at the end of the academic year or at the next Changeover.
	\item Once created, the Forum Working Committee is assigned to a VP and the VP selects head(s) just as they would with any other committee.
	\end{enumerate}
\end{enumerate}

\subsubsection{A-Team Search Committees}
	\begin{enumerate}[(a)]
	\item The purpose of an A-Team Search Committee is to make an informed recommendation to the Dean of Undergraduates as to who should fill a vacancy in the A-Team.  Additionally, the A-Team Search Committee is responsible for representing Duncan and Rice with appropriate etiquette and tact.
	\item An A-Team Search Committee is created at the discretion of the LVP to fill a vacancy or expected vacancy in the A-Team.
	\item There are two committee heads for a search committee.  The process of selecting the committee heads is as follows:
		\begin{enumerate}[(I)]
		\item The LVP runs a standard general election for one committee head position.
		\item The LVP shares the results of that election with the Executive Committee who, in consultation with the A-Team, selects a second head for the committee.  In the decision of the second head, consideration shall be given to ensure the two heads represent a reasonable level of diversity of the Duncan community.  To these ends, particular consideration shall be given to the spirit of this diverse representation in the scenario where an individual in consideration to serve as the second head is of the same academic class as the first head.  However, if the elected head is a member of the senior class, then the selected second head shall not also be a member of the senior class.
		\end{enumerate}
	\item No distinction is made between the two heads.
	\item The committee heads shall have the responsibility to select members of Duncan College to form the search committee.  The size of the committee, chosen by the committee heads, shall be large enough to reasonably represent diverse beliefs, mindsets, and interests of the college, yet small enough to meet and work efficiently.  It shall be preferred that the two heads select between four and six students to join the committee.  However, the heads may select fewer than four or more than six students if they receive approval from the College Master.  It is strongly preferred that the full committee contains representation from each of the senior, junior, sophomore, and freshmen classes.
	\item Once fully formed, the search committee is expected to solicit applications, collect information on the applicants, interview the applicants, and make an informed final recommendation.
	\item In the case that impeachment charges are brought against one of the committee heads, only the Master has the power to remove him or her.
	\end{enumerate}

\subsubsection{Representatives}
	\begin{enumerate}[(a)]
	\item All representatives must be defined in the same document as the committees (see ``Standing Committees").
	\item No representative may have power that overlaps with any office that is constitutionally defined.
	\item Representatives can either be elected or appointed.  That decision is up to the joint discretion of the VPs.
	\end{enumerate}


%***Other Student Offices***
\subsection{Other Student Offices}


\subsubsection{Orientation Week Coordinators}
\begin{enumerate}[(a)]
\item The O-Week Coordinators are responsible for organizing all aspects of Duncan's part of the University's O-Week.  They are also to assume all responsibilities expected of them by the University.
\item Duncan shall have two or three O-Week Coordinators who are selected each year by the previous O-Week Coordinators in consultation with the President and A-Team.
\item The O-Week Coordinators are expected to remain engaged with and be a point of contact for all matriculating students that were a part of the O-Week that they organized for the entirety of the corresponding academic year.
\item The O-Week Coordinators have the power to select advisors from applicants within Duncan and co-advisors from applicants who are members of other residential colleges.
\item O-Week Coordinators cannot be impeached by the process of the impeachment article in the Duncan College Constitution.  They can only be removed by the College Master.  If there is a vacancy in the O-Week Coordinator team, it can be filled by appointment from the Duncan College Master in consultation with the remaining members of the Coordinating team, the President, and the A-Team.
\end{enumerate}

\subsubsection{Technical Directors}
\begin{enumerate}[(a)]
\item The First Technical Director is in charge of managing all of the audiovisual needs of the college.  This includes, but is not limited to: setting up for public parties and other college events, managing the equipment, and training the next generation of Technical Directors.  He or she may also, on behalf of Duncan, charge a fee to non-Duncan groups for Duncan equipment and/or his or her services.
	\begin{enumerate}[(I)]
	\item New First Technical Directors are appointed by a vote of the Executive Committee.
	\item In order to be eligible to be appointed First Technical Director, one must have served as Second Technical Director.
	\item The EC's vote is to be informed by discussions with the outgoing First Technical Director, committee heads, and other parties that interact with the Technical Directors.
	\item There is not a term limit on First Technical Director.
	\item If the First Technical Director chooses, he or she may hold their position for multiple years contingent on a vote of confidence each year by the EC.
	\item The First Technical Director is overseen by a VP.
	\item The First Technical Director is a bump exempt position.  If the First Technical Director claims bump exempt status and then either resigns or is impeached, then he or she is automatically kicked off campus the subsequent year.  This can mean that an ex-Technical Director can be kicked for his or her senior year.  This consequence can only be lifted by the College Master under special circumstances.
	\end{enumerate}
\item The Second Technical Director(s) are responsible for assisting the First Technical Director in all of his or her duties.
	\begin{enumerate}[(I)]
	\item To be considered a Second Technical Director, one must submit a written statement to the First Technical Director and the Legislative Vice President by Winter Break of the academic year in which they serve.
	\item There is not a term limit on Second Technical Director.
	\item The Second Technical Directors are not bump exempt.
	\end{enumerate}
\end{enumerate}


%***A-Team***
\subsection{A-Team}

\subsubsection{Membership}
The A-Team is comprised of the College Masters, the Head Resident Fellows, the Residential Associates, and the College Coordinator.  These positions are defined and appointed by the University.

\subsubsection{Purpose}
The A-Team should serve to support  both the Duncan Government and the interests of the members of the college.  The A-Team also should serve as the primary liaison between the members of the college and the University administration.  The A-Team is not explicitly a piece of the Duncan Government, and only the Duncan College Master has any direct power over the Duncan Government.

\subsubsection{Duncan College Master}
The Duncan College Master is appointed to the college by the University and is subject to all of the expectations and responsibilities assigned to them by the University.  The purpose of the Duncan College Master is to look after both the health of the college as a whole and the individuals within it.  They are to serve as both a public advocate and the private voice of reason for the college.  The Master also serves as the head of the A-Team, and it is therefore acceptable to call the Master ``Hannibal" in all contexts formal or otherwise.

%*****Governmental Procedure*****
%*************************************
\section{Governmental Procedure}
%*************************************
%*****Governmental Procedure*****


%***Procedural Overview***
\subsection{Procedural Overview}


\subsubsection{Head of Procedure}
The Legislative Vice President is the procedural leader of the college and has discretion over all cases not specifically outlined in the Constitution.

\subsubsection{Voting Overview Table}

\begin{tabular}{| p{6cm} | l | p{6.5cm} | }
	\hline
	\bfseries{Issue}&\bfseries{EC Vote}&\bfseries{Forum Vote}\\
	\hline\hline\hline
	- Total Votes&- 6 and Pres. tie break&- 12 and Pres. tie Break\\
	\hline\hline
	\multicolumn{3}{ |c| }{\bfseries{Standard Vote}} \\
	\hline
	- Quorum&- 4, Pres. does not count&- 9, Pres. does not count\\
	- Proxies&- No proxies ever&- No more than 3\\
	- Motioning&- Motion and Second&- Motion and Second\\
	\hline\hline
	\multicolumn{3}{ |c| }{\bfseries{Constitution or Code of Conduct Vote}} \\
	\hline
	- Quorum&- N/A&- 12, Pres. does not count\\
	- Proxies&- N/A&- No more than two\\
	- Motioning&- N/A&- Motion and Second, once motioned and seconded there must be a minimum of 72 hours before the vote is held and a maximum of 192 hours.\\
	\hline
\end{tabular}

\subsubsection{Procedure of a Vote}
\begin{enumerate}[(a)]
	\item All votes require a motion and a second from two different voting members. When a motion is seconded, the motion must be recorded verbatim in the minutes for the meeting, along with the individuals who made the motion and the second. 
	\item The vote must be taken immediately unless there is a motion to amend the original motion or the Duncan Constitution otherwise requires a waiting period.
	\item If there is a motion to amend the vote, the motion for amendment must receive a second. There must be time for discussion of the proposed amendment before a vote regarding the amendment takes place. The amendment and the individuals who made the motion and the second for the amendment must also be recorded in the minutes. 
	\item If the amendment is passed, the vote will occur immediately on the newly amended motion.
\end{enumerate}

\subsubsection{Voting Responses}
\begin{enumerate}[(a)]
\item There are three ways in which a voting member may answer to a vote.  They may abstain or they may vote yea or nay.
\item Unless otherwise specified, all votes will be done by show of hands, starting with yea votes, then nay votes, then abstentions.
\item All votes of the EC and Forum are to be recorded by the LVP and made publicly available with the exception of the following votes of EC:
	\begin{enumerate}[(I)]
	\item Impeachments of appointed officials
	\item Votes of confidence to promote the Second Treasurer to First Treasurer
	\item Votes to select a First Technical Director
	\item Votes of confidence to retain the First Technical Director
	\end{enumerate}
\item If the two Secretaries or Treasurers disagree on how to cast the one vote of their office, then they should abstain.
\end{enumerate}


%***Functioning of Forum***
\subsection{Functioning of Forum}


\subsubsection{Voting Power}
\begin{enumerate}[(a)]
\item The Duncan Forum has the power to vote on all matters of the college and is considered the highest voting body within the college.
\item Any explicit decision made by any elected or appointed office, with the exception of the CJ enforcing the Code of Conduct, can be overturned by a vote of Forum.
\item Any decision by an elected or appointed official that a voting member of Forum deems to be both of consequence and a significant departure from precedent should be presented to Forum in such a way to give them the opportunity to vote on the matter.
\end{enumerate}

\subsubsection{Voting Members}
\begin{enumerate}[(a)]
\item The President (1 vote, only be cast in the event of a tie)
\item The two Vice Presidents (1 vote each)
\item The Treasurers (1 vote together)
\item The Secretaries (1 vote together)
\item The eight Class Representatives (1 vote each)
\end{enumerate}

\subsubsection{Proxies}
\begin{enumerate}[(a)]
\item In the event that a voting member of Forum is unable to attend a vote, he or she may send a proxy to cast a vote in his or her place.
\item The vote of a proxy is just as binding as the vote of the person in the actual position.
\item To be eligible to be a proxy, one must be a member of Duncan and neither a member of the Executive Committee nor a voting member of Forum.
\item In order for a person to be considered a proxy, either the President or LVP must have a written statement from the voting member designating his or her replacement.  This can be done by paper, e-mail, text message, or any other form the LVP or President deems acceptable.
\item At any one vote, there can be no more than 3 proxy votes.
\end{enumerate}

\subsubsection{Standard Vote}
\begin{enumerate}[(a)]
\item Any vote that is not seeking to change the Constitution or the Code of Conduct can be won simply by there being more yea votes than nay votes, unless otherwise specified.
\item For such a vote to be held, there needs to be a quorum of least two-thirds of the votes represented (9 votes) and at least half of the actual voting members (6 people).
\item The LVP shall moderate the vote.  In the event that the LVP is not present, the CJ shall moderate the vote.  If neither is present, a standard vote may not be held.
\end{enumerate}

\subsubsection{Votes on the Constitution or Code of Conduct}
\begin{enumerate}[(a)]
\item Any vote to amend either the Duncan College Constitution or the Duncan College Code of Conduct requires a three-quarters supermajority to pass.
\item For such a vote to occur, there must be a quorum of all 12 voting members, and there may be no more than two proxy votes among the 12. 
\item The President must be present for such a vote.
\item In this case, the President would never get a vote because a tie is irrelevant when a three-quarters supermajority is required.
\item The LVP shall moderate the vote.  The LVP must be present for a vote to amend either the Constitution or Code of Conduct.
\end{enumerate}


%***Functioning of the Executive Committee***
\subsection{Functioning of the Executive Committee}


\subsubsection{Voting Power}
The Duncan Executive Committee has the power to vote on all matters of policy that affect only the current academic year or their terms in office. Any vote that will have longer-term impacts on the college must be passed down to a vote at Forum.  The only exception to this is that all votes creating or modifying the budget must be passed down to Forum.

\subsubsection{Voting Members}
\begin{enumerate}[(a)]
\item The President (1 vote, only to be cast in the event of a tie)
\item The Chief Justice (1 vote)
\item The two Vice Presidents (1 vote each)
\item The Treasurers (1 vote together)
\item The Secretaries (1 vote together)
\item The Student Association Senator (1 vote)
\end{enumerate}

\subsubsection{Sending a Vote to Forum}
At the discretion of any single voting or non-voting member of the EC, any EC vote may be sent to a Forum vote.  Before each Executive Committee vote, there must be a designated moment for someone to make such a request.  This request cannot be made for designated EC votes such as confirmations for Treasurers, confirmations or appointments of Technical Directors, or impeachments of appointed officials.

\subsubsection{Winning a Vote}
All votes of the Executive Committee are won by a simple majority, unless otherwise specified.

\subsubsection{Proxies}
There can be no proxies for a vote within the EC.  In the absence of one of the Treasurers or Secretaries, the other Secretary or Treasurer can vote on behalf of the entire office.

\subsubsection{Quorum}
In order to hold a vote, the Executive Committee must have a quorum of four standard votes represented.  Therefore, the two Treasurers and Secretaries each only count as one for the purposes of a quorum.  The President's vote is a tie-breaker and therefore does not count towards the quorum.

\subsubsection{Moderator of the Vote}
The Legislative Vice President is tasked with moderating all votes of the Executive Committee.  In the absence of the LVP, the President may moderate the vote.  If neither is present, then a vote may not occur.


%***Functioning of the Court***
\subsection{Functioning of the Court}


\subsubsection{Initiation of Proceedings}
\begin{enumerate}[(a)]
\item Infractions may be reported to any Justice by any member of Duncan College, by the College Master, or by the Assistant Dean for Student Judicial Programs. Infractions reported by persons other than those listed above shall be turned in to the College Master, President, or CJ, who shall then file a complaint in the name of the person entering the complaint.  Complaints may be filed in the name of the college by the CJ or President.
\item Formal complaints may not be withdrawn.
\item The accused must be formally notified of the complaint within 72 hours of it being formally submitted.
\item Once notified, the accused has the right to designate counsel to help advocate for him or her and guide him or her through the process.  Any member of Duncan not on the Court or the Executive Committee, with the exception of the Legislative Vice President, may serve as counsel to the defendant if the defendant so chooses.  The accused may also act as their own counsel.  The primary duties of his or her counsel shall be:
	\begin{enumerate}[(I)]
	\item To orient the accused to the trial process.
	\item To explain to the accused his or her rights during the trial.
	\item During the trial, to point out facts in the defendant's favor of which the Justices do not seem to be aware or understand, but which the counsel deems important to the interests of the defendant.
	\end{enumerate}
\item Copies of all complaints and accusations shall be delivered to the College Master and the counsel of the accused before any trial proceedings are conducted.
\item The College Master shall be notified in advance of each Court action to ascertain whether information from the files of the accused is pertinent.
\end{enumerate}

\subsubsection{Hearing}
\begin{enumerate}[(a)]
\item The CJ should inform the accused student of the impending hearing and the details surrounding the accusation at least seven days before the hearing.  The accused counsel shall discuss the charges against the accused student, the evidence that has been collected at that point, the time and location of the hearing, and the rights of the accused.  If the accused student cannot attend the set trial hearing, the Court shall work with the accused counsel to set a time that works both for the student and for members of the Court.
\item A hearing shall be held within ten days of notifying the accused student of the charges, excluding University holidays and exam periods, unless for valid reasons postponement is agreed upon by the accuser, the accused, and a majority of the Court.
\item The accused shall have his or her rights explained to him or her in detail by his or her counsel.  Any questions regarding the hearing, the rights, or the accusation may be addressed through the counsel to the Chief Justice.
\item The trial shall be constructed so that the facts of the case may be efficiently obtained and a just decision reached while at no time violating the rights of the accused nor the integrity of the Court.
\item The Court may strike from the records testimony it deems irrelevant by unanimous vote.
\item Names of those not on the court involved in a hearing must remain confidential at all times to anyone who is not the accused, his or her counsel, and the members of the court.
\item At least five Justices must be present to open a hearing.  At least five of the Justices originally present must be present for the entire trial.
\item Members of the Court may recuse themselves from the trial.  If fewer than five Justices remain on the court, then a sufficient number of Class Representatives shall be selected by the CJ to maintain a five-member Court.
\item If the accused is a Justice of the Court, then he or she must recuse himself or herself from the hearing. If the Chief Justice recuses himself or herself, then the remaining Justices shall select one of their number to preside.
\item The accused may enter a plea of ``in violation" or ``not in violation" prior to the hearing.  Failure to enter a plea will be entered as a plea of ``not in violation."
\item Witnesses may be called or recalled by the accused or by the Court.
\item If, after proper notification, the accused fails to attend the hearing, the proceedings may be carried out in his or her absence. He or she forfeits all rights and may be tried accordingly.
\end{enumerate}

\subsubsection{Verdict}
\begin{enumerate}[(a)]
\item The verdict shall be determined immediately after the hearing in a closed meeting of the Court.
\item Each member of Court should individually determine wether the preponderance of the evidence indicates that a violation occurred.
\item A four-fifths vote of the Justices present during the entire hearing is necessary for a verdict of ``in violation."
\item If a verdict of ``in violation" is not reached by the Court, the accused is ``not in violation."
\end{enumerate}

\subsubsection{Sentence}
\begin{enumerate}[(a)]
\item Upon a plea or verdict of ``in violation," the Court will immediately determine the sanction of the accused in accordance with the Duncan Code of Conduct.
\item A majority vote of the Justices present for the entire hearing is necessary to determine a sanction or to reprimand the accused.
\item After a sanction has been determined and before it is implemented, the accused and the College Master shall be given a written notification of the action by the Court.
\item A sanction may be enforced only after the five day window to submit an appeal has closed.
\end{enumerate}

\subsubsection{Records}
\begin{enumerate}[(a)]
\item Records of the entire hearing shall be kept by the Chief Justice in a confidential file open only to the Justices, the Master, and any other applicable University officials.
\item Abstracts of hearings shall be kept by the Chief Justice and made available to the college members upon request.  Abstracts shall be posted publicly before the execution of sentences and shall remain posted for the period of one week.
	\begin{enumerate}[(I)]
	\item An abstract shall contain a statement of the accusation, the verdict, and the sentence handed down.
	\item An abstract shall not contain any personally identifiable details.
	\end{enumerate}
\end{enumerate}

\subsubsection{Appeals}
Appeals shall be handled in accordance with the Rice University Code of Student Conduct.

\subsubsection{Rights of the Accused}
\begin{enumerate}[(a)]
\item The accused has the right at any time to end the court proceedings and have their case transferred to the University Court.
\item The accused may dispute or review any testimony or evidence given.  In the case of witnesses called by the court, this can be presented in the form of the CJ's notes.
\item The accused has the right to be present, if he or she desires, when all evidence and testimony from witnesses he or she requests are presented.
\item The accused has the right to sum up the case before the Court decides the verdict.
\item The accused may call or recall witnesses; however, no character witnesses may be called.
\item The accused has the right to counsel.  The counsel must be a member of Duncan College or he or she may serve as his or her own counsel.
\item The accused has the right to be in contact with his or her counsel and have his or her counsel present at all times during the hearing and any other relevant meetings.
\item The accused has the right to request the Court to strike testimony from the records if he or she deems it irrelevant.  This requires a majority vote of the Court to pass.
\item The accused has the right to appeals as dictated by the Code of Student Conduct.
\item The accused has all the rights granted to them by the Code of Student Conduct.
\end{enumerate}


%***The Budget***
\subsection{Budget}


\subsubsection{Creation of the Budget}
\begin{enumerate}[(a)]
\item At the beginning of each academic year, the Treasurers are responsible for creating a budget proposal in consultation with EC and presenting it at Forum.  
\item Once presented, there must be a one week window in which any voting member of Forum can propose amendments.
\item The budget is to be passed as a standard vote of Forum.
\item In the interest of allowing the continuous functioning of the college, the Treasurers may authorize expenditures before the current year's budget is passed by parties that had line-item budgets in the previous year's budget .  These approvals must be disclosed to Forum.  Those early expenditures would then be deducted from the appropriate line-item once the budget is passed.
\end{enumerate}

\subsubsection{Management of the Budget}
\begin{enumerate}[(a)]
\item Forum may modify the budget with a standard vote at any time.  Only Forum may modify the budget.
\item The management of the budget is the responsibility of the Treasurers, and they have the power to define regulations and procedure for spending, reimbursing, and cataloging expenditures.
\item An up-to-date ledger of expenditures must be kept by the Treasurers and made available to any member of Duncan.
\item The VPs are responsible for monitoring the expenditures of offices they oversee to ensure that funds are being used for their intended purposes.
\end{enumerate}

\subsubsection{Regulations for the Budget}
\begin{enumerate}[(a)]
\item A budget must leave an amount in reserve (unallocated and not part of the operating budget) of at least 25\% of the amount allocated to Duncan from Rice in the previous year.
\item All line-items of the budget must include a list of persons authorized to spend from that sub-budget.
\item All line-items in the budget must satisfy one of these four criteria:
	\begin{enumerate}[(I)]
	\item Allocated to a party who directly reports to a member of the EC.
	\item Allocated to a member of the A-Team.
	\item Allocated to a sub-budget that requires a Forum vote to be utilized.
	\item An expense the college incurs in its natural functioning, such as: laundry machine rent, subsidy of stoles for graduation, and Powderpuff registration fees.
	\end{enumerate}
\end{enumerate}


%***Election Timing***
\subsection{Election Timing}


\subsubsection{Elections by Position}
\begin{enumerate}[(a)]
\item First Round Elections - The first round of elections shall be scheduled by the LVP between Winter Break and Spring Break with enough time to also run the second round of elections before Spring Break.  The following positions will be elected in the first round:
	\begin{enumerate}[(I)]
	\item President
	\item Chief Justice
	\item Legislative Vice President
	\item Student Association Senator
	\end{enumerate}
\item Second Round Elections - The second round of elections shall be scheduled by the LVP after the first round of elections in such a way that people who do not win in the first round may run in the second.  The second round elections must also be held such that the results are announced before submission of information to Room Jack.  The following positions will be elected in the second round:
	\begin{enumerate}[(I)]
	\item Vice Presidents
	\item First Treasurer (if the office is vacant)
	\item Second Treasurer
	\item Secretaries
	\end{enumerate}
\item Class Representative Elections
	\begin{enumerate}[(I)]
	\item The Class Representatives for the senior, junior, and sophomore classes for a given academic year shall be elected in an election scheduled by the LVP within the last four weeks of classes of the previous academic year.
	\item The Class Representatives for the freshman class shall be elected in an election scheduled by the LVP between the fourth and sixth week of classes of the academic year in which they will be holding office.
	\end{enumerate}
\end{enumerate}


\subsubsection{Changeover}
\begin{enumerate}[(a)]
\item The Executive Committee Changeover is a ceremony, scheduled by the President, at which offices of the Executive Committee are passed from the outgoing officers to the officers elect.
\item Changeover for the Orientation Week Coordinators is scheduled by the Orientation Week Coordinators.
\item Changeover for all other Duncan Government offices is at the end of the academic year.
\end{enumerate}


%***Election Code of Conduct***
\subsection{Election Code of Conduct}


\subsubsection{Entering an Election}
\begin{enumerate}[(a)]
\item All interested and potential candidates must submit a declaration of intent to enter a specific race.
\item Once a candidate has submitted his or her declaration of intent, he or she is considered entered in the race.
\item A candidate must adhere to all regulations listed within the Duncan Election Code of Conduct for any race he or she enters.
\end{enumerate}

\subsubsection{Candidate Regulations}
\begin{enumerate}[(a)]
\item Candidates may not post or advertise their candidacy on any electronic media or physical poster.
\item Candidates may advertise their candidacy by word of mouth but may not advertise to a captive audience.
\item Candidates may not engage in any offensive conduct specifically intended to hurt another candidate.  This includes but is not limited to, disparaging personal statements, offensive slander, and other direct personal or offensive assaults.
\item Candidates may not solicit votes from any person under the pretense of future political patronage or gift to that person.
\item Any candidate not in full compliance with the Duncan Election Code of Conduct is subject to removal from the candidacy by vote of the EC or, if elected, removal from office under the terms of impeachment of elected offices.
\end{enumerate}

\subsubsection{Determination of Elections}
\begin{enumerate}[(a)]
\item All election ballots shall be preferential and shall be counted accordingly.  This means that the ballot shall allow voters to rank all N number of the candidates in a given race from their first preference (1) to their last preference (N).  Ballots should be counted using the following procedure:
	\begin{enumerate}[(I)]
	\item Count all (1) votes.  If a candidate has more than half of the votes as (1) votes, then he or she wins.
	\item Eliminate the candidate(s) with the single fewest number of (1) votes.
	\item Promote all of the preference on the ballots from the eliminated candidate(s) until the (1) vote on each ballot is a non-eliminated candidate.  In the case of any tie, the candidate with the greater number of (1) votes on the initial ballots is the winner of the tie.
	\item Repeat the previous three steps until one candidate has a majority of (1) votes.  That candidate is declared the winner of the election.  
	\item For the Vice President election, where two winners must be chosen from a single race, the counting procedure is first run as it normally would be.  This will produce a single winner.  That person should then be eliminated and the votes should be promoted as though they had the fewest (1) votes.  The ballots should then be counted again using the procedure above.  This will produce a second winner.
	\end{enumerate}
\item The counting will be done by the Legislative Vice President and two tellers.  The tellers will witness the ballot counting and ensure accurate results.
	\begin{enumerate}[(I)]
	\item Tellers are chosen by the LVP and must be members of Duncan College in good standing with both Duncan College and the University.
	\item Tellers should be involved members of the Duncan community who are free of conflicts of interest.
	\item If the tellers find issue with the method of counting or the results of the count, they are to submit a statement to the Master and President.
	\end{enumerate}
\item The results of an election will be posted publicly within one day of being officially determined.
\end{enumerate}

\subsubsection{Petitions of Dissent}
\begin{enumerate}[(a)]
\item Any student may submit a petition of dissent or objection to the LVP up to one week from the results being announced objecting to the conduct of the elections, conduct of the candidates, or other concerns that would compromise the legitimacy and integrity of the college and the elections.  In the case that the complaint is brought against the LVP, it should be submitted to the President.  
\item The LVP, or President in the case that the complaint is brought against the LVP, will review the petition to ensure it is in the proper form before distributing it to the EC and Class Representatives.
\item Once petitions have been distributed, the Chief Justice will create a jury comprised of the Class Representatives with the Chief Justice presiding as a non-voting member to ensure proceedings are carried out justly.
\item The jury of Class Representatives will hear testimony from any persons they deem appropriate.  There will then be a discussion after which the following series of votes will be taken:
	\begin{enumerate}[(I)]
	\item ``Action should be taken on the election in question."  If this fails to pass, then the petition of dissent is discarded, the jury is adjourned, and the results of the election in question stand.  If it passes, then the jury should continue with the voting process.
	\item ``[Candidate] should be ejected from the election."  This vote must be taken on all candidates in the election in question.  If there are only two candidates and one is ejected, then the voting ends here and the remaining candidate is declared the winner.  If multiple candidates remain, then the election must be run again after following vote is taken by the jury:
	\item ``[The President, CJ, a VP] should take over re-running the election from the LVP."  The person running the election must agree to do so and must be free of conflicts of interest.  The person running the new election should commence the new election as soon as possible.
	\end{enumerate}
\end{enumerate}


%***Impeachment***
\subsection{Impeachment}


\subsubsection{Impeachable Officials}
Any official within Duncan can have impeachment charges brought against him or her by any member of the college if the official was elected by members of Duncan or appointed by a member of the EC, unless specified otherwise in the Duncan College Constitution.

\subsubsection{Requirements for Impeachment Charges}
A petition for impeachment may be brought against an official for exhibiting malfeasance, exhibiting extreme partiality, exceeding authority, or failing to meet vested responsibilities.

\subsubsection{Procedure for an Elected Official}
\begin{enumerate}[(a)]
\item A member of Duncan College must submit to the Legislative Vice President a formal petition of complaint detailing grounds for the impeachment of the elected official.  If the charges would be brought against the Legislative Vice President, then the petition is instead submitted to the Duncan President.
\item The Legislative Vice President (or President in the case the charges are brought against the Legislative Vice President) will review the petition to ensure it is in the proper form before informing the official against whom the charges will be brought.  He or she will then distribute copies of the petition to the Class Representatives.
\item Once petitions have been distributed, the Chief Justice will set a date for a trial with a jury comprised of the Class Representatives, with the Chief Justice presiding and non-voting to ensure proceedings are carried out justly.  In the event that the charges are brought against the Chief Justice, the Duncan President shall preside over the trial.
\item Before a petition of impeachment goes to a full hearing, the presiding body for the hearing must first hold a vote to ascertain the validity of the petition. If the body unanimously agrees that there is no merit to the petition based on the criteria for impeachment or that the petition lacks substantive evidence, the case will be thrown out and not be allowed to proceed to the full hearing. Any member of the body with a conflict of interest must recuse themselves from this discussion and the subsequent hearing. If the petition for impeachment passes this step, the hearing procedure will then begin.  This is done in order that petitions which do not address the ``Requirements for Impeachment Charges" or that are otherwise frivolous can be thrown out.  This is not a preemptive judgment on the guilt of the accused.
\item The jury will hear evidence and testimony from the accuser and will subsequently hear a defense from the impeached official. The jury will then have a period of time to investigate the matter through questions to the accuser, the impeached official, and any witnesses brought to the proceedings. After the jury has concluded investigation, the Class Representatives will adjourn to a private meeting to discuss the proceedings and arrive at a decision. The decision of the Class Representatives shall require a two-thirds majority to remove someone from office. The jury will immediately reconvene once the Class Representatives have reached a conclusion to announce the verdict of the jury. The decision of the Class Representatives shall be effective immediately after the adjournment of the jury.
\item It is the expectation that trials will be carried out as expeditiously as possible while remaining thorough.  Purposefully causing delays in a trial proceeding is in itself an impeachable offense.
\end{enumerate}

\subsubsection{Procedure for an Appointed Official}
\begin{enumerate}[(a)]
\item A member of Duncan College must submit to the Legislative Vice President a formal petition of complaint detailing grounds for the impeachment of the appointed official.
\item The Legislative Vice President will review the petition to ensure it is in the proper form before informing the official against whom the charges will be brought.  He or she will then distribute copies of the petition to the members of the EC.
\item The LVP will then take of a vote of the EC as to wheather the petition has validity.  If the EC unanimously agrees that there is no merit to the petition based on the criteria for impeachment or that the petition lacks substantive evidence, the petition will be thrown out.
\item Once petitions have been distributed, the President will convene the EC to deliberate over the impeached officials and vote upon their removal.
\item The EC will hear evidence and testimony from the accuser and will subsequently hear a defense from the impeached official.  The EC will then have a period of time to investigate the matter through questions to the accuser, the impeached official, and any witnesses brought to the proceedings.  After the EC has concluded their investigation, the EC will adjourn to a private meeting to discuss the proceedings and arrive at a decision. To remove an appointed official, the EC shall require a two-thirds majority.  The EC will inform the appointed official of their decision, which takes effect immediately thereafter.
\end{enumerate}

\subsubsection{Filling Empty Offices}
\begin{enumerate}[(a)]
\item A vacant office must be filled at the earliest time possible.
\item The same procedure is used to fill a vacancy as would have been used to originally fill it.
\end{enumerate}


%***Resignation***
\subsection{Resignation}


\begin{enumerate}[(a)]
\item Any appointed or elected office has the power to submit a resignation in order to remove themselves from office.
\item All resignations must be submitted to the President and LVP and include the name of the person resigning and the title of the position from which they are resigning.  In the event that the President or LVP is resigning, then the petition is just submitted to the other remaining office.
\item No resignation may be conditional or recanted.
\item Filling of an office that has been vacated due to a resignation shall be filled in the same way as though the officer was impeached.
\end{enumerate}


%***Room Jack***
\subsection{Room Jack}

Room Jack is the process by which Duncan allocates the limited number of on campus beds to its members.  This process naturally involves kicking some members off campus if the number of students requesting beds exceeds the number of available beds.

\subsubsection{Student Classification}
\begin{enumerate}[(a)]
\item Students shall be eligible to be kicked off campus if they are not bump exempt.
\item The class of a student shall be the graduation year of their matriculating class, with the exception of students who transferred into Rice University from another University. The class of a transfer student shall be their projected graduation year.
\item Bump exempt students shall be students who have declared their senior status with the college, new students, individuals who have lived off campus for at least one semester in the previous academic year, individuals who hold a constitutionally defined bump exempt Duncan Government office, and varsity scholarship athletes.  Studying abroad does not count as living off campus.
\item Senior status may only be declared once by a student and is automatically declared for a student's fourth year at Rice if it has not been previously declared.  A fourth-year student may, with permission of the Master, defer his or her senior status by up to one year if and only if he or she enters Room Jack for that year.
\item If someone is in his or her fourth year, but has not declared senior status (i.e. is planning on taking a fifth year and went through Room Jack as a junior) he or she will be classified as a junior.
\item Students who remain Rice students after declaring their senior status shall be classified as post-senior students and shall not be eligible to enter Room Jack.
\item A student may be classified as bump exempt under special allowance if they satisfy one of the following:
	\begin{enumerate}[(I)]
	\item Documented medical consideration that requires that a student live on campus.
	\item Documented medical consideration in which a student's quality of life would be significantly lower living off campus solely due to the medical consideration.
	\item Extenuating circumstances that compromise a student's safety outside of the normal risks associated with living and being a Rice student.
	\end{enumerate}
\item All conditions for a student to be classified as bump exempt under special allowance must be thoroughly substantiated to the Master, who then communicates the condition satisfied to the Legislative Vice President.
\item Any student who receives a room from Room Jack but decides to relinquish the room shall be classified as bump exempt in the subsequent year if and only if another student takes his or her place.  For the student to receive a bump exempt status in the subsequent year, the room may not be left vacant.
\end{enumerate}

\subsubsection{Procedure for Room Jack}
\begin{enumerate}[(a)]
\item The procedure for Room Jack shall be determined by the Legislative Vice President.  The procedure must be accessible to all current students, be random, and be demonstrably fair and unbiased.
\item The Legislative Vice President will post at least two weeks in advance of Room Jack the procedure to be used for Room Jack.
\item Any student failing to submit all required forms for Room Jack shall not be allowed to enter Room Jack.
\item Failing to enter Room Jack shall result in automatically being kicked off campus.
\item Room Jack shall be conducted publicly at a time most students can attend.
\item Students shall be kicked off from the rising-sophomore class and rising-junior class.
\item The number of students kicked from each class shall be equal in proportion to the size of each of the classes from which students may be kicked.
\end{enumerate}

\subsubsection{Waitlist}
\begin{enumerate}[(a)]
\item The waitlist must have a hierarchy constructed in the following way:
	\begin{enumerate}[(I)]
	\item Students kicked off in Room Jack (randomly ordered).
	\item Students who elected to move off campus and wish to instead live on campus for the following year (ordered by time-stamp of official communication with the LVP).
	\end{enumerate}
\item  The waitlist must be accessible to all Duncan students.
\end{enumerate}


%***Room Draw***
\subsection{Room Draw}

Room Draw is the process by which Duncan assigns room numbers to students who have passed through Room Jack and received on campus housing.

\subsubsection{LVP Relocation}
The room assigned to a student during Room Draw is a request, not a right.  The LVP has the power to change a student's room assignment both before and after the student has moved into the room.  This should only be done with good reason and in consultation with the Master or College Coordinator.  A student may appeal a relocation to the College Master.

\subsubsection{Room Draw Order}
\begin{enumerate}[(a)]
\item Room Draw will be organized in such a way that there is sufficient time for a group who fails to receive a room in a given round of draw to reorganize and enter the next round.
\item 1st Draw: President
	\begin{enumerate}[(I)]
	\item The President has first pick of any one of the draw categories listed below and may populate the remaining beds with anyone he or she chooses so long as those people were not kicked off campus.
	\item The President does not have the power to bring people back who were kicked off campus in Room Jack or who chose to go off campus.
	\end{enumerate}
\item 2nd Draw: 5-Man Closed Suites (5th B\&C)
	\begin{enumerate}[(I)]
	\item The two 5-man suites on the 5th floor will be drawn second.  This is done in order to give groups that fail to draw either 5-man suite the opportunity to go in for 6-man suite draw.
	\end{enumerate}
\item 3rd Draw: 6-Man Closed Suites (2nd - 5th A\&D)
	\begin{enumerate}[(I)]
	\item The eight closed 6-man suites will be drawn third.
	\end{enumerate}
\item 4th Draw: Open Suites (2nd - 4th B\&C)
	\begin{enumerate}[(I)]
	\item Open suites may be drawn as groups of six or eight people, effectively giving each group the option to include the two singles outside the suite proper in their draw.
	\item There is no preference in draw order as a result of whether the group is six or eight people.
	\item All singles in open suites that are not drawn by groups will go into singles draw. This also includes the remaining two singles from any six person group that receives an open suite.
	\end{enumerate}
\item 5th Draw: Singles
	\begin{enumerate}[(I)]
	\item All of the singles not classified as suites, as well as any remaining singles from the open suites draw, are drawn fifth.
	\end{enumerate}
\item 6th Draw: Doubles
	\begin{enumerate}[(I)]
	\item The new O-Week Coordinators will communicate with the LVP and reserve doubles for new students.  The reserved new student rooms must be distributed more or less evenly throughout all of the halls on second through fourth floors.  The LVP and O-Week Coordinators may choose to negotiate and change which rooms are reserved but are under no obligation to do so.  All non-reserved rooms are drawn in doubles draw.
	\item In the event that more people want doubles than are available, an extra round of singles draw will be held after doubles draw.
	\end{enumerate}
\end{enumerate}

\subsubsection{Order that Students Select Rooms}
\begin{enumerate}[(a)]
\item The order in which people are given the chance to draw their rooms is based on a point system rooted in seniority.  In this section every class is referred to by its rising status.  Therefore, Room Draw is only concerned with sophomores, juniors, and seniors.
\item The point values are assigned as listed below:
	\begin{enumerate}[(I)]
	\item Seniors = 3
	\item Juniors = 2
	\item Sophomores = 1
	\item Any student who failed to receive their Room Draw Point = 0
	\end{enumerate}
\item How the Draw Tiers Function
	\begin{enumerate}[(I)]
	\item Every person brings a point value to a group; all of the point values of people going into draw together are averaged for a single group value.
	\item Once groups are sorted into tiers, the groups in each tier are put into a randomized order to select rooms. This randomized order must be done by the LVP in a way that is demonstrably fair and unbiased.
	\end{enumerate}
\item Designation of the Room Draw Tiers
	\begin{enumerate}[(I)]
	\item People/Groups with a value $\ge$ 2.66 (Senior draw)
	\item People/Groups with a value $\ge$ 1.66 (Junior draw)
	\item People/Groups with a value $\ge$ 0.66 (Sophomore draw)
	\item People/Groups with a value $\ge$ 0 (Failed to meet Room Draw Quota)
	\end{enumerate}
\end{enumerate}


%***Parking Jack***
\subsection{Parking Jack}


Parking Jack is conducted to allocate the North College Lot spaces that the University Parking Office allocates to Duncan College.  Parking Jack is run by the LVP.

\subsubsection{Eligibility}
Any full-time, non-graduating student and member of Duncan College is eligible to enter Parking Jack.  The requirement of being a full-time student can be waived by the Master.

\subsubsection{Order of Preference}
\begin{enumerate}[(a)]
\item The jack will be organized into tiers to determine preference.  The order of the tiers are listed below:
	\begin{enumerate}[(I)]
	\item The President
	\item All rising seniors
	\item Rising juniors living off campus for their junior year.
	\item Rising juniors living on campus for their junior year.
	\item Rising sophomores living off campus for their sophomore year.
	\item Rising sophomores living on campus for their sophomore year.
	\end{enumerate}
\item The people within each tier will be put into a random order.
\end{enumerate}


%***Freshmen Service Points***
\subsection{Freshmen Service Points}


\subsubsection{Freshmen Classification}
\begin{enumerate}[(a)]
\item For the purposes of Freshmen Service Points, freshmen shall be defined as students who matriculated to Rice in the fall semester of that academic year without having completed a freshman year at another university or college.
\item A student will also be given freshman status if he or she matriculated at Rice in the fall of the academic year in question and he or she plans to complete four years at Rice.
\end{enumerate}

\subsubsection{Acceptable Service Point Opportunities}
\begin{enumerate}[(a)]
\item In order for Service Points to be awarded, the opportunity must have been made equally available to all freshmen simultaneously.
\item No party offering Service Points may give preference to its own members or any group when offering Points.
\item Freshmen Service Points may not be awarded for any opportunity that involved any selection process other than one based on the time they indicate willingness to participate.  The only exception to that are opportunities that require certifications openly available from Rice University (e.g. Server Certification, Caregiver Training). 
\item Service Points must directly benefit Duncan College. Service to the greater Rice, Houston, American, or worldwide communities is encouraged but will not be counted for Freshmen Service Points.
\item The college must offer a variety of Service Point opportunities. These opportunities must include some that are not related to alcohol and some that are offered during the daytime. Failure to seize these opportunities will not be regarded as an excuse for failure to meet the point requirement.
\end{enumerate}

\subsubsection{Establishing the Point Quotas}
\begin{enumerate}[(a)]
\item There are two quotas for Service Points.  The first quota is called the Room Jack Quota, and the second is called the Room Draw Quota.
\item In collaboration with the VPs and committee heads, the LVP shall establish the point quotas for that year before the second week of classes of the academic year in question.
\item The quotas must be designed so that there is ample opportunity for all freshmen to fulfill both quotas by the time at which they are enforced.
\item If, for any reason, the point quotas do not reflect the number of points being offered, the LVP, in consultation with the EC, may lower one or both quotas.
\item Once set, quotas may not be raised.
\end{enumerate}

\subsubsection{Registering for Service Point Opportunities}
\begin{enumerate}[(a)]
\item The registration for all service opportunities must be made available to the college at least a week before the event.  This can be overridden in extreme cases by the Legislative Vice President.
\item In the event that a student registers for a service opportunity and then fails to satisfactorily complete that service as determined by the party offering the opportunity, he or she will lose the same number of points he or she otherwise would have gained.  It is therefore possible to have negative points.
\item At discretion of the LVP, a student may unregister for a service point opportunity.  To unregister, a student must give advance notice to the LVP and the party offering the opportunity with sufficient time to find a replacement.
\end{enumerate}

\subsubsection{Failure to Meet the Room Jack Quota}
\begin{enumerate}[(a)]
\item In the event that a student fails to meet the Room Jack Quota, he or she will not receive on campus housing his or her sophomore year.
\item Failure to meet the Room Jack Quota will not negate any other bump exempt status.
\end{enumerate}

\subsubsection{Failure to Meet the Room Draw Quota}
\begin{enumerate}[(a)]
\item A student may never go into Room Draw with more than zero points if he or she has not completed their Service Points.
\item If a student fails to meet his or her Room Draw Quota during his or her freshman year, he or she may complete those points in any subsequent year to earn housing status consistent with his or her class.
\item As soon as a student fulfills his or her Room Draw Quota, he or she will go into all future rounds of Room Draw with the appropriate number of points for his or her class as listed in the Room Draw section.
\end{enumerate}

\end{document}